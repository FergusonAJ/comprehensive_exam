\section{Introduction}

Chapters \ref{chap:replaying_associative_learning} and \ref{chap:varying_environments} investigate the role of historical contingency in Avida. 
However, do the trends seen in Avida translate to other digital systems? 
This is the question we ask in this chapter.  

When writing all but the simplest software, we are plagued by making countless design decisions. 
Designing digital evolution systems is no different. 
How should organisms store data? 
Which instructions or operators do we include?
How large should the initial genome be, and what should it contain?
Is the genome directly encoded?
These are all valid questions when designing an evolving system, and each has multiple justifiable answers. 
However, with each a design decision comes a branching point in how that system operates. 
Depending on the specific decision, the difference between two choices can have a drastic effect on how evolution proceeds. 

Avida is no exception to this predicament; there are countless design decisions in Avida that affect the different dynamics of evolution. 
While some of these aspects have been empirically tested \citep{ofriaDesignEvolvableComputer2002, brysonUnderstandingEvolutionaryPotential2013}, others are nearly impossible to investigate because one small changes sends ripples throughout the other parameter spaces. 
%As an example, as seen when investigating evolved genomes in Chapter \ref{chap:alife_submission}, large portions of the genome can be un-expressed, resulting in a large fraction of mutations being neutral. 
%How does this interact with historical contingency? 
Here I propose we investigate the role in representation in potentiation by subjecting an alternate digital substrates to evolution and analysis on the same task. 
We leverage the patch harvesting task from Chapter \ref{chap:varying_environments}, whose inputs and outputs can easily be encoded to the various data types needed. 
I will evolve Markov brains \citep{hintzeMarkovBrainsTechnical2017} on the same environment, looking at the potentiation of mutli-patch harvesting behavior. 
%I will then analyze replicates from each substrate that successfully evolved memory usage. 
%Did successful lineages follow pathways similar to Avida? 
%How does potentiation change across substrates? 
By comparing these results to those in Chapter \ref{chap:varying_environments}, we can conduct a direct comparison in how potentiation changes as we change representations. 

My hypothesis is that the details of potentiation changes will vary as we change the underlying representation, but the general trends will stay the same. 
Markov brains are also capable of epistatic interactions, and I suspect that mutations that set up a potential interaction for the future will again cause large changes in potentiation. 
As mutations are quite different in the two systems, I expect to see difference in what these mutations are actually doing. 
For example, every mutation in Avida either changes, adds, or deletes a single instruction, while in Markov brains a mutation can change not only the type of a gate, but also how the other genes are interpreted.
Additionally, Markov brains have memory built in; some outputs are actually memory values that are then available as inputs in the next update. 
As such, I do \textit{not} expect Markov brains to potentiate with mutations that prep memory like we might see in Avida. 

% I expect substantial differences across substrates, however, I expect some general trends to hold true regardless the system used. 
% Specifically, I expect single mutations to greatly increase potentiation in each of the substrates. 
% Further, I expect to see the same general potentiation patterns (see discussion of Chapter \ref{chap:alife_submission}) across substrates. 
% By moving beyond Avida, we can begin to see general trends in potentiation and historical contingency in digital systems, and indeed, begin to set expectations for natural systems. 