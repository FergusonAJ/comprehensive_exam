%%%%%%%%%%%%%%%%%%%%%%%%%%%%%%%%%%%%%%%%%%%%
%%%%%%%%%% COMPS PROPOSAL OUTLINE %%%%%%%%%%
%%%%%%%%%%%%%%%%%%%%%%%%%%%%%%%%%%%%%%%%%%%%

% Introduction
    % Dicsuss digital evolution substrates 
        % How are they similar?
        % How are they different?
    % Discuss differences in how they interact with evolution
        % Stuff like the comparative hybrid approach
% Methods
    % This will be very similar to the two other replay chapters
    % Still not sure if we should build off of associative learning or patch harvest
        % Associative learning is the more complex behavior, but how does it translate between substrates?
        % Patch harvesting is simpler but can still result in fairly complex behavior
            % Cues are MUCH simpler (~4 instead of 10^6 possibilities)
            % Advanced strategies still require memory, however
    % Substrates 
        % Avida (same as before)
        % Markov brains
        % RNNs?
        % CGP?
        % AvidaGP?
    % Otherwise, collection will be roughly the same as in Chapter 3 and 4
    % For the first time, we will be comparing across substrates, so we need to be able to statistically compare them
        % Distributions of Potentiation patterns
        % Phylogenetic distance between potentiation and target behavior
        % Largest potentiation jumps and dips