\section{Outlook}

In this dissertation proposal I have outlined my plans to study how history interacts with adaptation and chance to produce cognitive behaviors. 
Specifically, I first demonstrated the power of analytic replay experiments \textit{in silico}, and I proposed to expand these studies to create the first large, cross-lineage dataset of how potentiation changes along successful lineages. 
Next, I proposed shifting to a simpler bitstring model to further develop our intuition and analysis techniques for potentiation. 
Taking a step back, I examined how the evolution of phenotypic plasticity influences future evolutionary dynamics, finding that plasticity stabilizes evolution. 
Finally, I proposed to study potentiation in two additional environments to test the generalizability of my earlier findings as the study system changes. 
%Specifically, the first two chapters demonstrate the power of analytic replay experiments \textit{in silico} by generating the first large cross-lineage dataset of how potentiation changes along successful lineages. 
%By examining this potentiation of associative learning, I can begin to identify general trends in how potentiation arises and any properties shared by mutations that influence it. 
%Further, these data will provide a point of comparison for future studies of potentiation, whether those be \textit{in silico} or \textit{in vivo}.
%Next I proposed a 

This proposal originally had one more chapter: an alternative proposal to study potentiation in different representations. 
Specifically, I would repeat the potentiation study in the patch harvesting environment using Markov brains, recurrent neural networks, and Cartesian genetic programming. 
By keeping the environment the same, this would demonstrate how potentiation dynamics change with the representation.  
While this chapter has been cut for brevity, it remains an option if a proposed chapter encounters unsalvageable issues or if my committee prefers it to another chapter. 

%Overall, this work asks important questions about the interplay of adaptation, chance, and history in the evolution of particular traits. 
The nature of this work is mostly exploratory; prior work has measured potentiation in specific systems, but not enough work has been conducted to begin drawing general conclusions about potentiation. 
This is the gap I aim to fill. 
While all the work I have proposed is digital, by studying different environments and representations, I will create intuition about potentiation that is likely to extend beyond digital systems.
Doing this work \textit{in silico} is required, however, as the work that I have proposed is currently intractable to conduct \textit{in vivo}.
Regardless, here I expand on the previous work on microbial potentiation to help us understand how adaptation, chance, and history interact to create the diverse complexity we see in the world. 

There are countless ways the work in this proposal could be extended. 
Since this area is so new and so difficult to study in wetlab systems, quantifying potentiation after varying \textit{any} aspect of the system can provide new insights into how potentiation arises and the forms that it takes. 
For example, while the cyclic logic 6 environment of Chapters \ref{chap:consequences_of_plasticity} and \ref{chap:varying_environments} changes over time, a more systematic study of the effects of temporal changes on potentiation would be relevant for many natural systems, where both biotic and abiotic factors are constantly changing. 
I am particularly interested the fundamentals of potentiation, and as such I am curious how potentiation changes in other simple models like the NK landscapes of Chapter \ref{chap:simplified_model}.
Specifically, I am interested in the effect of switching from bitstrings to multi-allele representations. 
When targeting only single trait in a bitstring model, every single-bit mutation must bring you either closer or farther away from that trait. 
By switching to a multi-allele model, we could analyze mutations that are neutral with respect to the genetic distance to the target. 
Finally, the analyses here focus on target traits that are effectively optimal, but this is not required. 
How do potentiation dynamics change as we switch to targeting suboptimal traits, and how does the potentiation of one trait affect the potentiation of other traits?
These ideas only begin to scratch the surface of possibilities in studying potentiation dynamics, and it will be exciting to see how this area of evolutionary biology develops over the coming decades. 

% [I keep wanting to explain why these chapters are repetitive. That's the point! I'll be running the same experiment 3 or 4 times with slightly different setups to give us initial data to compare how potentiation changes. Did this get across? I don't think so]

% In writing this proposal, I have had a few nagging concerns in the back of my mind and want to discuss them briefly here. 

% First and foremost, the four proposed chapters feel repetitive and lacking in literature depth. 
% This issue has bothered me throughout writing this document. 
% At the end of the day, I have come to realize that my main goal in those chapters was to show how 


% In writing this dissertation proposal, I cannot help but think that it feels very different than those that I have read before from friends and colleagues. 
% Last year at my first committee meeting, I mentioned that I was working on what is now \ref{chap:alife_submission}, but that most of my time had been spent implementing the basics of Avida in MABE2. 
% This has greatly shaped the second half of my PhD. 
% While re-implementing Avida, I conducted research, but most of it does not fit this theme and is thus is not included in this dissertation proposal. 
% As such, it feels like I am proposing an insane amount of work, though the proposal chapters themselves feel repetitive. 
% I would like to speak on that briefly here. 

% In writing this dissertation proposal, I the proposed chapters were repetitive and lacking in literature depth. 
% I am a slow writer, which means that I have sat with these thoughts for quite a while, and I have some thoughts. 
% The chapters are repetitive due to the nature of what I have proposed in this document. 
% Proposals should include two things: the work you will be doing and why it matters. 
% In both cases, these do not change much between the potentiation chapters. 
% The only variation in methods is how I will be varying the environment or representation, while the positioning in the literature is constant across chapters. 

% Beyond the literature and methods, all proposed chapters have effectively the same end goal: to generate initial data so we can begin to hypothesize about how potentiation occurs in different system, and if any general rules or patterns exist. 
