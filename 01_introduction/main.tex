\chapter{Introduction}
\label{chap:intro}
% Add section about fitness landscape and search space topology stuff
% Add section that defines all keywords 
%  - add to chapter 1 and will be many section 
%  - explanation of terms and how we are using them
% Add section describing different ea branches (EA, ES, GP, GA)
% 

% overall thesis and the first half of what a selection scheme is
% This thesis focuses on conducting research to enhance the understanding of evolutionary algorithms by developing a theoretical framework to better define them and enhance the abilities of evolutionary algorithms by developing new diagnostic tools that reveal strengths and weaknesses.
% Evolutionary algorithms are an optimization procedure inspired by biological evolution for real-world problems and many flavors exist, such as genetic programming [CITE], genetic algorithms [CITE], evolutionary programming [CITE], and evolutionary strategies [CITE].
% While each flavor of an evolutionary algorithm possesses distinct characteristics that make it unique and applicable to certain problem domains, similarities can be found within each of them.
% Traditionally, evolutionary algorithms follow three key phases: evaluation, selection, and variation. 
% The framework developed in this thesis formally defines the selection scheme of an evolutionary algorithm, a component typically used during the selection phase of an evolutionary algorithm, into three components: population structure, trait construction, and selectors.
% Indeed, this framework allows us to apply tractable changes to a selection scheme and measure the impact the change has on problem-solving successes. 
% Additionally, this framework allows practitioners to easily identify the selection scheme within an evolutionary algorithm, while also helping reduce the likelihood of constructing redundant selection schemes.

% In Chapter 2, I focus on formalizing the selection scheme of an evolutionary algorithm, and in Chapters 3 and 4, I demonstrate how small alterations to a scheme can lead to different performances on the same optimization problem.
% In particular, lexicase selection [CITE] is the selection scheme that is extended through the lens of our selection scheme framework and assessed in Chapters 3 and 4.

% Ultimately, our new design of lexicase selection proved to be more effective than standard lexicase selection.




% When new evolutionary algorithms or components are developed, they are typically assessed through benchmark suites that contain optimization problems from different domains. 
% The latter allows practitioners to better understand the impact of the choices made for the configuration of an evolutionary algorithm has on its problem-solving success.
% Currently, the standard approach to evaluate an evolutionary algorithm's capabilities is through benchmark suites with differing optimization problems from different domains [CITE].



As many evolution-focused dissertations start, there is seemingly limitless diversity to the organisms that have evolved in nature. 
From microbes to megafauna, evolutionary processes have created a stunning array of traits and behaviors in organisms. 
But how did these features come to exist?
And what general evolutionary trends can we abstract from these examples? 
These, of course, are grand challenges of evolutionary biology. 

In addressing these challenges, biologists naturally examine the fossil record for clues on how evolution produced life as we know it.
Looking to history, however, has many limitations: we are provided with only a single instance of evolution, the data we do have is incomplete, and we are not able to go back in time to conduct controlled experiments.
In recent decades there has been a surge in experimental approaches to studying evolution.
If we evolve populations in controlled laboratory conditions, we are able to evolve many populations in parallel, observe almost everything that occurs, and build a range of experimental conditions -- thus addressing each of the problems above.
%If we evolve populations in controlled laboratory conditions, we are able to ask evolutionary questions that are [] with the fossil record alone. 
%By doing X, Y, Z, we can do blah. 
%Recent decades have seen a surge in ``experimental evolution'' -- observing evolution \textit{as it happens}, typically in a laboratory or digital setting. 
%In this work, it is through this lens that we will tackle evolutionary questions. 
%In this dissertation, I will conduct this research using an experimental mindset toward understanding historical contingency.
In this dissertation, I will adopt this experimental mindset in an attempt to understand the role of historical contingency in evolution. 
%conduct this research using an experimental mindset toward understanding historical contingency.
I will use the possibilities revealed from counterfactual experiments to understand "life as it could have been" in our study systems.
This techniques will help me separate the flukes from the inevitabilties in the dynamics that shaped the course of evolution as it was originally realized in these studies.

When looking at evolution in nature, we often encounter beneficial traits that were uniquely evolved in one type of organism, while organisms in the broader taxonomic unit found different survival strategies. % in a way that others did not. 
%When looking at evolution in nature, we often encounter highly beneficial traits that are restricted to the evolution of one or a few species. evolved in one species, while other species found different survival strategies. 
For example, while many species of birds and insects exhibit self-powered flight, only one branch of mammals have: bats. 
Why was this behavior so rare, and what conditions led to the evolution of flight in only this one specific branch of mammals? 
In nature, there is only so much we can do; unfortunately we cannot travel back in time and ``replay the tape of life'' (as evoked by \citet{gouldWonderfulLifeBurgess1990}) to observe if flight consistently evolved in bats, or if it was an uncommon stroke of luck.
Could we, instead, use experimental evolution to ask similar questions?
As I describe below, biologists have successfully employed this approach to understand the evolution of microbial traits, and I seek to further refine these techniques.

In thinking about the evolution of a particular trait, it is important to consider three different factors: adaptation, chance, and history \citep{travisanoExperimentalTestsRoles1995}.
%Adaptation clearly plays a role, as traits that provide a benefit to the survival and reproduction of an organism are more likely to be carried on to the next generation and eventually increase in frequency. 
New or modified traits that provide a net benefit to the survival and reproduction of an organism are more likely to increase in frequency, thus illustrating the role that \textit{adaptation} plays in shaping evolutionary outcomes. 
%Was there a clearly evolutionary pathway where the trait was selected?
%Evolution is an inherently stochastic process. 
Of course, adaptation can only act upon genetic sequences that are available in the population.
The stochastic nature of random mutations means that some genetic sequences will arise, while others will never even appear for selection to consider in the first place.
The random appearance of sequences -- or disappearance as misfortune can remove otherwise fit genotypes -- highlights some of the roles of \textit{chance} in evolution.
%Due to the stochastic nature of random mutations and genetic drift, however, we cannot fully predict evolution in even simple systems. 
%Did this stochasticity play an important role?
%Did a fluke mutation or unlikely allele sweep make all the difference?
%This is the contribution of chance.
The influence of chance is constrained, though, by the starting genotypes that mutations act upon; traits can appear or be modified only if such changes are available in the local genetic neighborhood.
%Finally, we must consider history. 
%In genetic space, the traits in close proximity heavily depend on where you start. 
%The traits that can be mutated in depend on where you are in genetic space.
%As such, small, seemingly insignificant mutations can have drastic downstream effects. 
%Thus, the \textit{history} that has resulted in the current distribution of sequences in the population significantly effects what might evolve.
This distribution of genetic sequences in the population at the point under investigation in an evolutionary study can be described as the product of its \textit{history}.
It is the interplay of these three aspects that produce the complex dynamics that we observe in evolving populations.
%Did such a mutation matter in the evolution of the trait we care about? 
%What might that mutation have looked like? 

%TODO: move this to somewhere appropriate
Of course, what we call history is just a matter of temporal perspective.
From the vantage point of a population in a given state, all of the dynamics that brought the population to that state are now all consolidated under the label of history.
Looking forward, however, both adaptation and chance will be at play, with different mutations or combinations of mutations occurring at different probabilities, and the resulting combinations having differing survival potential.
As time advances, these changes are again relegated to history, while new outlooks now exist for the population based on its new composition.
As such, for any study where we examine the balance among these three factors, we need to be clear about the starting point from which our perspective will be based.

In this work, I focus on the balance between chance and adaptation and how that balance changes over the course of history.
From one point along a lineage, the evolution of a given trait may be unlikely, subject to the whim of chance.
From a later point along that lineage, that same trait's evolution may shift to being a near certainty, with adaptation more in control of the population's fate.
Is there some way for us to predict these shifts in influence between adaptation and chance?
Can we distinguish between a population that is being driven to a specific outcome from one that is simply adrift?
And how do these shifts occur?
Does chance give way to adaptation in small increments, or can an individual mutation dramatically alter the balance?

The challenge with addressing these problems using standard experimental evolution techniques is that they require us to have speed, control, and data collection capabilities beyond what is currently possible.
Specifically, we must be able to isolate all of the individual mutations along a lineage and replay evolution using each step as a new starting point.
Furthermore, for each of these starting points, we need to be able to conduct enough replays to generate statistically powerful conclusions about evolutionary outcomes.

%By leveraging experimental evolution, not only can we observe traits as they evolve, we also have access to experimental control unthinkable in natural systems. 
I leverage digital evolution to overcome these hurdles, , which gives many benefits over wetlab approaches, including greater speed, automatic high resolution data collection, and the ability to start an experiment with the exact conditions of our choosing.
%three main benefits: 
%(1) we can observe traits and populations \textit{as they evolve} with perfect accuracy, 
%(2) we are given access to experimental controls unthinkable in natural systems, 
%and (3) we can evolve populations with unprecedented speed. 
Of course, digital evolution also has its drawbacks. 
For example, there is a much lower limit to the complexity of the organisms and what we have been able to evolve \textit{in silico}.
Further, due to technological constraints, digital evolution is typically limited to scales of at most tens of thousands of organisms, while natural populations can be vastly larger.
%Digital evolution is limited in scale by technological constraints that do not effect populations evolving in nature. 
%While the scale of populations and their interactions in nature can produce unthinkable complexity, digital evolution is limited by technological constraints. 
As such, evolving meaningfully complex traits from scratch in open-ended systems requires a better understanding of the underlying dynamics to be able to maximize evolvability. 

While there are many complex traits that could be studied, here I focus on the evolution of early cognitive behaviors. 
This topic is of great interest to evolutionary biologists in understanding the origins of intelligent behaviors.
The lack of obvious physical characteristics of intelligence makes it challenging to study the evolution of these traits by looking at the fossil record, and their complexity makes them difficult to re-evolve under laboratory conditions.
%Here, I focus on one such area: the transition from purely reactive phenotypically plastic organisms to those capable of very basic cognition. 
%Of course, this domain has also proven challenging for evolving digital systems, as artificial intelligence turns out not to be a solved problem.
Of course, this domain has also proven challenging for evolving digital systems, which is of little surprise as artificial intelligence as a field has encountered many hurdles in its quest to produce intelligent agents. %, varied history of artificial intelligence shows that creating agents with intelligent behaviors is no easy task.
Investigations into the origins of simple stepping stones to intelligence, however, have been much more fruitful. 
%While these behaviors have been evolved, this evolution is often either rare or in very specific systems tuned to the task. 
By looking at the role of history in the evolution of early cognitive behaviors, I aim to shed some light on how these behaviors arise, why they are so difficult to evolve, and how we might increase the complexity of intelligent behaviors that digital evolution systems can produce.
\section{The role of history in evolution}

% Background - history vs chance vs adaptation
%Evolutionary biologists have long debated the factors that contribute to evolution. 
While the ideas of evolution and natural selection have been around for well over one hundred and fifty years \citep{darwin1859}, evolutionary biologists continue to argue about, test, and expand upon the different factors that contribute to evolution. 
Here I focus on the role of history in evolution, one of the three aspects succinctly identified by \citet{travisanoExperimentalTestsRoles1995}. 
%How these particular aspects of evolution interact has been a source of long standing debate. 
While adaptation was the initial frontrunner, researchers argued for the importance of chance % (in the form of random mutations and genetic drift) 
\citep{kimuraEvolutionaryRateMolecular1968, king1969non, mayrHowCarryOut1983} and later history \citep{gouldSpandrelsSanMarco1979, gouldWonderfulLifeBurgess1990} in evolution. 
% More recent examples? What does the field think now?

% What do we mean by the role of history in evolution? 
%In talking about the history of evolution, 
%It may appear obvious that history plays an important role in evolution, as evolution inherently relies on new organisms coming from those that already exist. 
It may appear obvious that history plays an important role in evolution, as the set of genetic sequences that could feasibly appear in the population relies on what sequences currently exist.
Here, however, I mainly focus on the idea of ``historical contingency'' -- the idea that small, often initially inconsequential changes can have a drastic effect on what ultimately evolves. 
As an example, consider a set of three genes, A, B, and C, that together give rise to a highly beneficial trait. 
All three genes are equally beneficial in isolation, while AB is slightly more beneficial but AC and BC are both detrimental to fitness. 
In this scenario, populations that have fixed either A or B in isolation have a beneficial pathway to the combined trait, ABC. 
If instead a population has fixed C by itself, the ABC trait becomes much harder to evolve, as both intermediate steps are deleterious; a double mutation is then needed to reach the trait without losing fitness. 
Even in this simple example, the initial fixing of C has no penalty when it occurs, but it shifts the possibilities of what is likely to evolve in the future. 
For a thorough review of the ideas and complications of historical contingency, as well as empirical investigations into its role in evolution, see \citep{blountContingencyDeterminismEvolution2018}.

While work has been done to study the role of historical contingency in the evolution of natural populations (e.g., \citep{lososContingencyDeterminismReplicated1998, kellerHistoryChanceAdaptation2008a}), here I base my work on empirical studies of historical contingency in experimental evolution.
Early work in \textit{Escherichia coli} produced two drastically different results. 
Researchers found no influence of the initial value of fitness reflected in the populations' final evolved fitness, but in the same experiment, they found that the final evolved cell size of a population was highly contingent in the initial cell size of that replicate \citep{travisanoExperimentalTestsRoles1995}.%one trait (fitness) saw no change at the end of evolution regardless of the population's initial value, while another evolved trait (cell size) did vary depending on the starting condition \citep{travisanoExperimentalTestsRoles1995}. 
Building off this framework, \citet{flores-moyaEffectsAdaptationChance2012} found evidence that history plays a key role in the evolution of growth rate and toxin cell quota in algae. 
Recently, \citet{smithFitnessEvolvingBacterial2022} have shown that, while history does play a role in the evolution of \textit{E. coli}, the interactions between adaptation, chance, and history can heavily depend on traits under investigation and the environment being studied. 
By leveraging clever experimental evolution studies, these researchers have shown that it is possible to disentangle the influence of adaptation, chance, and history on evolution in a particular system. 

% For a review of empirical \textit{in vitro} studies on historical contingency, see \citep{blountContingencyDeterminismEvolution2018}.
% This currently feels _very_ brief

Digital systems have also been used to study historical contingency's role in evolution. %, often leveraging the strengths of digital systems to conduct experiments not possible in living systems. 
By comparing normal evolutionary replicates to those where deleterious mutations were automatically reverted, \citet{covertiiiExperimentsRoleDeleterious2013} found that initially-deleterious mutations can increase the complexity of evolved traits. 
Separately, \citet{yedidHistoricalContingentFactors2008} found that, using a controlled extinction event, pre-extinction presence of a complex trait factored into the re-evolution of the trait after the event. 
The seminal work of \citet{travisanoExperimentalTestsRoles1995} was also replicated in the digital evolution system Avida and expanded to look at the contributions of adaptation, chance, and history over time, thanks to the perfect record keeping of digital systems \citep{wagenaarInfluenceChanceHistory2004a}.
More recently, \citet{bundyHowFootprintHistory2021} leveraged the speed of digital evolution to test the how the depth of history affects future evolution, dealing with generation counts well beyond what is currently feasible in microbial systems. 
Finally, \citet{braughtEffectsLearningRoles2007} recreated the initial \textit{E. coli} and Avida experiments using neural networks and demonstrated an interaction between adaptation, chance, history, and the Baldwin effect; they found that learning can influence the impact of the three factors. 
These studies show that not only are these techniques viable in digital systems, but that digital systems can \textit{expand upon them} to conduct research that would otherwise be impossible.

\subsection{Potentiation}

While this dissertation proposal focuses on investigations of the role of history in the evolution of cognitive behaviors, most chapters emphasize the concept of ``potentiation''. 
Here we define potentiation as the likelihood that a target trait evolves from a given initial genotype or population.

The foundational work for measuring potentiation comes from \citet{blountHistoricalContingencyEvolution2008}. 
The authors empirically tested whether the novel citrate metabolism in one of the Long Term Evolution Experiment populations \citep{lenskiLongtermExperimentalEvolution1991} was due to a fluke mutation or the accumulation of a potentiated genetic background. 
To do so, they founded multiple ``replay'' populations from various points along the lineage that originally evolved to metabolize citrate. 
They found that the metabolization of citrate was more likely to evolve from samples further along the lineage, providing support that genetic potentiation was a key factor. 

Essentially, this framework is applying the analysis of \citet{travisanoExperimentalTestsRoles1995} along each step of a lineage, and by varying the amount of evolutionary history present, we can identify shifts in the contributions of adaptation and chance in the evolution of the target trait.
Increases in potentiation indicate an increased contribution of adaptation, as the trait is now more likely to evolve. 
This could happen if the population has moved such that a more-adaptive (or less un-adaptive) pathway to the target trait now exists. 
Decreases in potentiation, on the other hand, indicate a stronger reliance on chance and could be the result of convergence to a local optima in the fitness landscape.
The selective pressure of this optima could leave the population reliant on fluke mutations or genetic drift to escape and potentially find the target trait.
Ultimately, a difference in potentiation between two points on a lineage indicates that the genetic changes between them, which are considered history in the context of the later point, are important in whether the target trait ultimately evolves.
This opens the possibility of examining what mutations fall in this window, how they affected the organism overall, and if their appearance in the lineage was due to adaptation or chance.


Since that initial study, researchers have conducted similar experiments (now called analytic replay experiments) in various systems and looking at various traits. 
These include evolvability in \textit{E. coli} \citep{woodsSecondorderSelectionEvolvability2011}, novel receptor usage in Phage $\lambda$ \citep{meyerRepeatabilityContingencyEvolution2012}, and, recently, the epistatic interactions in yeast \citep{vignognaExploringLocalGenetic2021}.
Across these systems, the researchers showed that the accumulated genetic background is profoundly important in the eventual evolution of the target trait. 
While these techniques are relatively new, they offer valuable insight into how the interplay of adaptation, chance, and history can influence what subsequently evolves. 
These studies look backward, empirically testing what changes led to the final evolved behaviors, but this is deeply intertwined with concepts such as predictability in evolution. 
As such, I expect research in the near future to begin weaving these findings into the broader tapestry of evolutionary dynamics.
%For further examples, \citet{blountContingencyDeterminismEvolution2018} provide a review of studies that perform analytic replays and other similar experiments that examine the role of historical contingency in evolution. 

% Studies of potentiation provide a glimpse into the impact that the accumulated genetic background had on later evolution. 
% This is deeply intertwined with the idea of predicting evolution, and in fact will likely heavily influence how we view evolutionary predictability in the future. 
% So far, however, potentiation studies have looked backward, empirically testing what changes led to the final behaviors that eventually evolved.  
% [TODO: Finish this thought]
\section{The evolution of cognitive behaviors}

% What do we mean by cognitive behaviors?
Previous studies of historical contingency in digital organisms have primarily explored the evolution of Boolean logic functions \citep{wagenaarInfluenceChanceHistory2004a, bundyHowFootprintHistory2021}.
%Here I instead focus on the evolution of cognitive behaviors, as they can be more intuitive to identify and to understand the stepping stones that lead up to them. % as compared to reactive behaviors. 
%Here I instead focus on the evolution of cognitive behaviors, as they can be more intuitive to understand and to identify the stepping stones that lead up to them. % as compared to 
Here I instead focus on the evolution of cognitive behaviors, which are more intuitive to understand as phenotypic traits, though their internal mechanisms can still be opaque.

Cognition focuses on sensing external information, dynamically processing it, and using the results to select a behavioral response.
The specific definition of cognition is debatable, but all examples of cognitive behaviors in this work use past experiences to make more effective choices (i.e., they require memory).
%The specific definition of cognition is debatable, but all examples of cognitive behaviors in this work require memory in order to make optimal choices. 
As such, these behaviors are integrating information over time, and I argue that clearly categorizes them as cognitive. 
%The information used and the type of processing performed determine the category of a cognitive behavior.
%For example, if the processing of environmental inputs is due to a rigid genetic encoding, it is usually described as phenotypic plasticity.
%By definition, cognitive behaviors make use of external information 
%In categorizing these behaviors, 
%I define ``reactive behaviors'' as those that require information about the current state of the environment in order to make an optimal decision, but do not require memory of past events.
%This still allows for complex genetically-encoded behaviors, such as the phenotypically plastic regulation of metabolism that I will discuss in Chapter \ref{chap:consequences_of_plasticity}. 
%Conversely, cognitive behaviors require more than just the current state of the environment. 
While many interesting behaviors fall under this cognitive umbrella, in this dissertation I focus on two of the most simple: (1) remembering environmental cues and associating them with optimal behaviors (Chapters \ref{chap:alife_submission} and \ref{chap:replaying_associative_learning}), and (2) monitoring local resource availability to identify when to shift between feeding on the current nutrient patch and searching for a new patch (Chapter \ref{chap:varying_environments}). 
%While an astounding number of behaviors fall under this umbrella, here I focus on some of the most simple: using one bit of memory to switch between two states (Chapter \ref{chap:varying_environments}) and associating a environmental stimuli with behavior (Chapters \ref{chap:alife_submission} and \ref{chap:replaying_associative_learning}). 

%While this includes many higher level intelligent behaviors, we such as memory or higher-level information processing.% (e.g., integrating over multiple sensors). 

% Why focus on them here?
%Why focus on cognitive behaviors? 
A myriad of cognitive behaviors exist in animals, and debatably many exist in plants and microbes as well \citep{loyWhereAssociationEnds2021, dussutourLearningSingleCell2021a}. 
%there is debate on whether the most simple forms of cognition (e.g., habituation, associative learning) are found in plants and microbes \citep{loyWhereAssociationEnds2021, dussutourLearningSingleCell2021a}.
Evolving these behaviors \textit{in silico} is therefore critical if we want to create useful agents or to study more complex evolutionary dynamics found in nature. 
However, evolving cognitive behaviors in digital systems has traditionally been difficult. 
It is a challenge worth pursuing, though, and replay experiments to disentangle historical contingency provide a new opportunity to make progress.
%It is a challenge worth pursuing, though, and as such it is a good choice for studying historical contingency in evolution. 
%Cognitive behaviors have the potential for complicated effects from historical contingency. 
At the same time, the complex nature and multiple required components of cognitive behaviors create a valuable scenario for deepening our understanding of historical contingency. 
As an example, a mutation that provides an organism with the capacity for memory may be initially deleterious (or at best neutral) if the machinery to utilize that memory is not in place.
%However, the existence of that memory also makes it more likely that the machinery needed to use it will be selected if it appears.
However, if that machinery were to appear, the lack of that memory might render it useless.
It is only in combination that these two traits form a beneficial behavior.
As such, either these traits must arise simultaneously for adaptation to be able to act upon the combination, or one must persist by chance until the other provides it with utility.
%It is only with the combination of the two that we would expect [X] to persist. 
%As such, we would only expect the two to persist 
%However, the existence of that memory greatly increases the benefit of the machinery needed to use it in the case that it does appear, increasing selection pressure and increasing the likelihood that the mutation is not immediately lost to drift. 
These possibilities raise the question: In lineages that successfully evolve cognitive behaviors, do we see an ``all-or-nothing'' simultaneous evolution of multiple interacting components, persistence of one component without benefit, or other dynamics such as exaptation of other traits?
The work I propose here will illuminate critical steps in the evolution of cognitive behaviors, providing useful information for future attempts to evolve them while also establishing a framework to ask larger questions about the interplay of adaptation, chance, and history. 

% What's so hard about evolving cognitive behaviors?
% Sidenote: we are starting from scratch, no memory baked in or anything like that
As mentioned, evolving these behaviors \textit{in silico} can be a monumental challenge.
I wish to make two key notes. 
First, there have been many studies focused on the interplay of learning and evolution (for a historical example, see \citep{hinton1987learning}), but here we are solely focused on evolving cognitive behaviors and not the downstream effects after cognition appears.   
The interactions between learning and evolution have long been theorized and studied \citep{baldwinNewFactorEvolution1896}, and this broad area of literature is generally outside the scope of this dissertation proposal. 
Second, it is important to note that here we are evolving these behaviors \textit{from the ground up}, with little to no built-in machinery to assist in the evolution of cognition. 
Many representations such as Markov brains and recurrent neural networks have aspects like memory built in \citep{hintzeMarkovBrainsTechnical2017}. 
These representations and the work that has been done with them are invaluable, but here we start at a low level, requiring even simple building blocks like memory to be evolved. % along with the rest of the organisms. 
%As such, evolving even the most basic cognitive behaviors is an uphill battle, but we are thus able to studying these dynamics 
While every digital system must make assumptions and use abstractions in designing the framework of organisms, I argue that requiring memory be evolved moves us closer to the challenges faced by early organisms in nature. 
Ultimately, these dynamics will need to be studied under a broad range of conditions and representations in order to draw generalized conclusions.

\subsection{Challenges in the evolution of cognitive behaviors}
% Why is it hard?

% Things to mention: 
%   - Bootstrapping problem
%   - Deceptive landscapes and local optima

% Deceptive landscapes
One common hurdle in evolving cognitive behaviors is one familiar to all researchers in evolutionary computation: deceptive fitness landscapes \citep{lehmanOvercomingDeceptionEvolution2014, whitleyFundamentalPrinciplesDeception1991, silvaOpenIssuesEvolutionary2016}. 
Here, we refer to fitness landscapes as genotype-to-fitness maps %(usually in a low dimension to aid in visualization), 
and deception as local optima that prevent evolution from reaching the target trait or global optimum. 
Deceptive landscapes are an issue in many areas of evolutionary computation, but they become especially problematic when evolving cognitive behaviors. 
The local optima that cause the issue are often behaviors that do not use memory, but still manage to do well enough to dominate a population \citep{risiEvolvingPlasticNeural2010}. %, preventing the reaches of genetic variation from discovering the cognitive behaviors. 
These local optima restrict the exploratory capabilities of the population and prevent the discovery of the target cognitive behaviors, even if they would otherwise be superior if given the opportunity. %perform better and would be selected should they appear. 
For example, bet-hedging techniques will often arise where organisms stochastically choose between two behaviors; if picking the correct behavior half of the time is sufficient for a net boost in fitness, such strategies will dominate.

% Bootstrapping
Further, the evolution of cognitive behaviors suffers from the ``bootstrap problem'', where no positive fitness gradient exists between initial conditions and genotypes that exhibit cognitive behaviors \citep{mouretOvercomingBootstrapProblem2009, gomezIncrementalEvolutionComplex1997, silvaOpenIssuesEvolutionary2016}. 
While this dissertation proposal argues for the importance of history in evolution, there is no denying that selection is a powerful driver of evolution. 
As such, an issue arises when stepping stones to cognitive behaviors are often not advantageous and thus not selected when they first appear. 
As described above, the capacity for memory will only be useful in conjunction with machinery that makes use of the stored information.  
%In the example mentioned above, the capacity for memory itself is not useful; it only becomes useful when combined with machinery to utilize it.  %unless the organism can pull from that memory in a beneficial way. 
Such situations are especially common with cognitive behaviors, where several components are all required to click into place all at once for any of them to be useful.
Each such instance makes the final behavior exponentially less likely to evolve. %, which can be extremely unlikely if not impossible in practice. 
In these cases, additional evolutionary incentives must often be employed to bootstrap the necessary building blocks to eventually reach these behaviors, as I describe below.
%It it is worth noting that this is a similar problem faced by researchers in the evolution of modularity, and future work should look to that literature for inspiration \citep{wagnerRoadModularity2007, cluneEvolutionaryOriginsModularity2013}.

\subsection{Previous work}

% What's been done overall?
The challenges inherent to evolving cognitive abilities in digital systems have encouraged researchers to develop various approaches to overcome them.
Researchers have augmented the fitness function of organisms to reward them for memory usage or other indicators of cognitive abilities; this has been met with success in neural networks solving T-mazes \citep{ollionLittleHelpSelection2012} and in Markov brains integrating over time \citep{schossauInformationTheoreticNeuroCorrelatesBoost2016}. 
These approaches fall under the general umbrella of behavioral decomposition, where organisms are independently evaluated on multiple aspects of a task.
Variations in this idea can be seen in evolving the learning process separately from memory \citep{nordinEvolutionWorldModel1998} or evolving distinct components to solve subtasks \citep{duarteHierarchicalEvolutionRobotic2012}.
Instead of evolving multiple components, some researchers have found success evolving a single system that is tested in progressively more difficult environments \citep{gomezIncrementalEvolutionComplex1997}.
These incremental evolution approaches are not a panacea, however, and have been demonstrated failing at improving the evolution of cognitive behaviors \citep{christensenIncrementalEvolutionRobot2006}. 

Beyond incremental evolution, others have argued that due to the deceptive nature of the fitness landscapes in these problems, one method is to abandon the objective, whether wholesale or to some lesser degree, and instead to encourage the exploration of novel behaviors. 
This has been demonstrated in the neuroevolution of memory usage \citep{lehmanOvercomingDeceptionEvolution2014}.
Additionally, \citet{carvalhoCognitiveOffloadingDoes2016} demonstrated that, while we often think of reactive, non-cognitive behaviors as local optima that hinder the evolution of cognitive behaviors, in some circumstances they can be effective stepping stones instead.


% What about Avida specifically?
Most of the work in this proposal build upon the Avida digital evolution framework \citep{ofriaAvidaSoftwarePlatform2004a}, which was previously used to study the evolution of cognitive behaviors. 
\citet{grabowskiEarlyEvolutionMemory2010a} demonstrated that Avida organisms can evolve rudimentary memory in a simple path-following environment. 
In a special case, this path following task even saw the evolution of counting in an odometric strategy \citep{grabowskiCaseStudyNovo2013}.
\citet{pontesEvolutionaryOriginAssociative2020} expanded on this work to show that organisms can evolve to associate random nutrient cues with the different turning directions, an early form of associative learning. 
It is off of the foundation these works that I conduct Chapters \ref{chap:alife_submission} and \ref{chap:replaying_associative_learning} of this proposal. 

% General outlook on improving the evolution of them (or save this for a conclusion in the real dissertation?)

% Conclude the background section
In this work, I aim to uncover trends in the role that history played in the evolution of these cognitive behaviors. 
After identifying which mutations played key roles in making the evolution of the behavior inevitable along a lineage, we will analyze those mutations in greater detail.
Ideally, we may even be able to leverage this information in future attempts at evolving the behavior to increase our ability to target specific complex traits. 
Additionally, by looking at potentiation across different environments and genetic representations, we can better understand how the decisions made about our experiment (e.g., how to structure the environment and what representation to use) alter the ultimate probability of successfully evolving the target behavior. 
\section{Completed and proposed work}
%\newcommand*{\theadaltb}[1]{\multicolumn{1}{c}{\bfseries #1}}

%This dissertation proposal investigates how the contributions of adaptation, chance, and history change over time in the evolution of cognitive behaviors. 
In this section I provide a breakdown of what is covered in each of the remaining chapters.
Table \ref{tab:chapter-guide} is provided for an overview at a glance.

% \setlength{\tabcolsep}{16pt}
% \renewcommand{\arraystretch}{1.5}
% \begin{table}[ht]
%     \centering

%     %\rowcolors{2}{gray!25}{white}
%     \begin{tabularx}{0.9\linewidth}{lXXX} % p{10cm}
%         \rowcolor{gray!50}
%         \hline
%         \theadalt{Chapter} & \theadalt{Representation}  & \theadalt{Environment} & \theadalt{Focus} \\
%         \hline
%         \rowcolor{gray!25}
%         2 + 3 & Avida & \makecell[l]{Associative \\ learning} & Potentiation \\
%         \rowcolor{white}
%         4 & Bitstring & NK landscapes & Potentiation\\
%         \rowcolor{gray!25}
%         5 & Avida & Cyclic logic 6 & \makecell[l]{Evolutionary \\ consequences} \\
%         \rowcolor{white}
%         6 & Avida & \makecell[l]{Cyclic logic 6 + \\ Patch harvesting} & Potentiation\\
%         \rowcolor{gray!25}
%         7 & Markov brains & Patch harvesting & Potentiation \\
%         \hline
%     \end{tabularx}

%     \caption{An overview of the study system and focal evolutionary dynamic for each chapter.}
%     \label{tab:chapter-guide}
% \end{table}

\newcolumntype{b}{X}
\newcolumntype{s}{>{\hsize=.5\hsize}X}
\newcolumntype{x}{>{\hsize=.45\hsize}X}
\setlength{\tabcolsep}{16pt}
\renewcommand{\arraystretch}{1.5}
\begin{table}[ht]
    \centering

    %\rowcolors{2}{gray!25}{white}
    \begin{tabularx}{\linewidth}{|xxbss|} % p{10cm}
        \rowcolor{gray!50}
        \hline
        \theadalt{Chapter} & \theadalt{System}  & \theadalt{Environment} & \theadalt{Focus} & \theadalt{Status} \\
        \hline
        \rowcolor{gray!25}
        2 & Avida 5 & \makecell[l]{Associative  learning} & Potentiation & Submitted \\
        \rowcolor{white}
        3 & Avida 5 & \makecell[l]{Associative  learning} & Potentiation & In progress \\
        \rowcolor{gray!25}
        4 & Bitstring & NK landscapes & Potentiation & Proposed \\
        \rowcolor{white}
        5 & Avida 2 & Cyclic logic 6 & \makecell[l]{Evolutionary \\ consequences}  & Published\\
        \rowcolor{gray!25}
        6 & Avida 5 & \makecell[l]{Cyclic logic 6 + \\ Patch harvesting} & Potentiation & Proposed\\
        %\rowcolor{white}
        %7 & Markov brains & Patch harvesting & Potentiation  & Alternate\\
        \hline
    \end{tabularx}

    \caption{An overview of the study system, focal evolutionary dynamic, and current status  for each chapter in this dissertation proposal.}
    \label{tab:chapter-guide}
\end{table}


% We start with \textbt{Chapter \ref{02_alife_submission}}, a current submission that analyzes the potentiation of associative learning. 
As in many experimental evolution studies, we can run multiple replicates and count how many evolve a specific behavior.
In \textbf{Chapter \ref{chap:alife_submission}}, I start to investigate the question: As an individual replicate progresses, can we identify if the evolution of a target behavior has become either impossible or inevitable?
I focus on retrospective analyses of four successful lineages in Avida, measuring the likelihood that associative learning ultimately re-evolves when restarting from each step (\textit{i.e.}, I track \textit{potentiation} over time).
%This is what I study in \textbf{Chapter \ref{chap:alife_submission}}, a current submission that analyzes changes in potentiation along lineages that successfully evolved associative learning in Avida. 
I find that potentiation can increase suddenly, even with a single phylogenetic step.
%Leveraging analytic replay experiments, I examined four case study lineages and found that potentiation can increase drastically in a single phylogenetic step. 
These potentiating mutations are hard to pin down, however, as some mutations are clearly related to associative learning while the effects of other mutations remain unclear. 

I propose to extend this work in \textbf{Chapter \ref{chap:replaying_associative_learning}}, expanding well beyond four case-study lineages, collecting more comprehensive data, and attempting to draw statistically powerful conclusions.
%%While Chapter \ref{chap:alife_submission} demonstrated %the effectiveness of the system and showed 
%that potentiation can occur suddenly, limiting the study to only four lineages proved restrictive when looking at \textit{how} the mutations were potentiating. 
%Therefore, in Chapter 3 I propose to replay many more lineages and analyze them in the aggregate. 
By extracting summary statistics about the changes in potentiation, we can identify patterns %in this measure and develop more informed hypotheses 
and collect more data
about the different types of mutations shown to promote potentiation.  %that might potentiate a particular trait. 
Additionally, these potentiation measurements will provide a basis of comparison for future work on this topic, both in this proposal and beyond.

But what are the underlying mechanisms for a mutation to increase potentiation?
%Even before conducting the work for Chapter \ref{chap:replaying_associative_learning}, the results from Chapter \ref{chap:alife_submission} have provided evidence that there are multiple ways a mutation might increase potentiation.
While some mutations may simply move toward the target trait in genotype space, others appear to move \textit{away} from that behavior while still increasing potentiation. 
%We have found evidence that some potentiating mutations directly introduce the focal behavior to the local fitness landscape, others shift the local fitness landscape so there is a pathway between the current genome and the focal behavior (though the behavior itself is not in the immediate landscape), and finally some potentiating mutations increase the benefit of the focal behavior so it becomes more likely to be selected. 
In \textbf{Chapter \ref{chap:simplified_model}}, I will shift to a more tractable system to fully explore these possibilities, while also testing the generality of my earlier results.
Specifically, I will quantify potentiation in bitstrings evolving on NK landscapes. 
%Simplifying the model gives us several benefits: A) it allows us to run many more replicates in less time, B) it allows us to enumerate spaces in a way that is impossible in more complicated models, and C) it is easier to fully analyze portions of the landscape. 
In addition to providing greater speed and expanded analysis possibilities, analyzing NK landscapes will allow us to examine the relationship between epistatic interactions and potentiation. 
As such, this simplified model will provide the first glimpse into how potentiation dynamics change as we vary representations and environments, putting the associative learning results in a broader context and setting the stage for generalized hypotheses of potentiation.
Furthermore, NK landscapes can be made small enough to allow exhaustive analysis of potentiation across the entire landscape, not just along a single lineage.
These additional data will allow me to conduct more comprehensive analyses of potentiation dynamics.

%\textbf{Chapter \ref{chap:consequences_of_plasticity}} is previously-published work that takes a step back to ask what downstream effects can occur \textit{after} a trait evolves.
\textbf{Chapter \ref{chap:consequences_of_plasticity}} is previously-published work that examines phenotypic plasticity and whether or not it potentiates associated traits.
%asks what downstream effects can occur \textit{after} a trait evolves.
%analyzes what happens to evolution \textit{after} adaptive phenotypic plasticity evolves. 
I analyzed the effects of reactive phenotypic plasticity, a common stepping stone for cognitive behaviors, on future evolution. 
I found that adaptive plasticity stabilizes new tasks once they evolve, but does not seem to increase the probability of them evolving in the first place. % the de novo origin of the other tasks under investigation.
Specifically, in a fluctuating environment, plasticity shifts the evolutionary dynamics (evolutionary change, retention of novel tasks, deleterious mutation accumulation, etc.) closer to those of a static environment. 
While this chapter does not directly investigate cognitive behaviors, it does indicate that plasticity increases potentiation through stabilizing new traits.
Furthermore, it indicates that lineages that evolve reactive plasticity before evolving cognitive behaviors may benefit from similar stabilizing dynamics.

%These methods of looking at the role of history in evolution in non-trivial environments are not limited to associative learning. 
%I then propose \textbf{Chapter \ref{chap:varying_environments}} to use the cyclic environment from Chapter \ref{chap:consequences_of_plasticity} and a patch harvesting environment to ask if patterns in potentiation generalize across environments.
In \textbf{Chapter \ref{chap:varying_environments}}, I ask how patterns in potentiation generalize across environments.
I will do this by comparing the potentiation of associative learning in Chapter \ref{chap:replaying_associative_learning} and in NK landscapes in Chapter \ref{chap:simplified_model} to two new environments: the evolution of optimal plasticity in the cyclical environment (from Chapter \ref{chap:consequences_of_plasticity}) and a cognitive multi-patch harvesting behavior. 
%While the potentiation dynamics in a single environment are interesting in their own right, finding cross-environment similarities in how potentiation changes would provide strong evi
If we see similar trends in potentiation across these four environments, that would support the hypothesis that potentiation follows predictable dynamics, that are likely to generalize to natural systems. %potentially expanding out to living systems. 
%This combination of environments was chosen to evaluate if a shared complex feature, in this case memory usage in the associative learning and patch harvesting environments, will cause similar potentiation dynamics.
% Why does this matter?
% This work builds off of Chapter \ref{chap:simplfied_model}, varying only the environment and not the representation, to disentangle the effects of the two


%In \textbf{Chapter \ref{chap:varying_environments}} I outline in-progress work we are doing on the evolution of memory use in patch-harvesting organisms. 
%While simple patches can be consumed by purely reactive processes, more complex environments (e.g., those with multiple patches) require basic memory to consume. 
%Specifically, we expect that organisms that consume multiple patches must be able to swap between two states: consuming a patch and finding the next patch. 
%Did these lineages evolve simpler behaviors before evolving memory? 
%What steps potentiated the evolution of memory, and is this similar to associative learning in Chapter 3? 
%These are the answers my collaborators and I aim to answer. 

%Chapters \ref{chap:replaying_associative_learning} and \ref{chap:varying_environments} use the Avida Digital Evolution Platform, but that is not necessary for investigating potentiation. 
% \textbf{Chapter \ref{chap:varying_representations}} is an alternative proposal to Chapter \ref{chap:simplified_model}, in which I ask the simple question: do our previous findings on potentiation generalize beyond Avida? 
% While Chapter \ref{chap:simplified_model} investigated this question using bitstrings on an NK landscape, here I propose to replicate the study of potentiation of memory usage in patch harvesting from Chapter \ref{chap:varying_environments} using Markov brains instead of Avidia organisms.
% %Specifically, how does potentiation change when we switch from Avida to Markov Brains?
% If we see similar patterns in potentiation across representations, that will be powerful evidence of general trends in how potentiation changes along a lineage. 
% If we do \textit{not} see general patterns, that will provide evidence that, if patterns in potentiation do exist, we need to take a step back and think about them at a higher, more abstract, level.
% Regardless of the result, expanding our studies beyond Avida is an important step in studying potentiation, and it will provide a solid foundation upon which future studies can build upon. 

Chapter \ref{chap:timeline} includes my final thoughts on this dissertation proposal. 
I discuss my perspective on where this work fits into the existing literature and where it might lead in the future.
Additionally, dissertations are only accepted if they are finished, %and as such it is important to have a plan for when I will complete the work that I am proposing here. 
%Therefore, 
so this chapter also includes a proposed timeline for when the various components of each chapter will be conducted.

%All together, this thesis proposal aims to identify patterns in how potentiation changes over a lineage. 
%Regardless of if we are able to identify general patterns in potentiation, this work will provide additional framework for thinking about, discussing, and testing potentiation as well as a basis for future comparative studies.
% \section{Genetic programming}
% \section{Genetic algorithms}
% \section{Evolutionary programming}
% \section{Evolutionary strategies}
% \section{Similarities}
% \input{01_introduction/tex/selection-schemes}
% \input{01_introduction/tex/thesis-work}