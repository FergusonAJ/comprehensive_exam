\chapter{Introduction}
\label{chap:intro}
% Add section about fitness landscape and search space topology stuff
% Add section that defines all keywords 
%  - add to chapter 1 and will be many section 
%  - explanation of terms and how we are using them
% Add section describing different ea branches (EA, ES, GP, GA)
% 

% overall thesis and the first half of what a selection scheme is
% This thesis focuses on conducting research to enhance the understanding of evolutionary algorithms by developing a theoretical framework to better define them and enhance the abilities of evolutionary algorithms by developing new diagnostic tools that reveal strengths and weaknesses.
% Evolutionary algorithms are an optimization procedure inspired by biological evolution for real-world problems and many flavors exist, such as genetic programming [CITE], genetic algorithms [CITE], evolutionary programming [CITE], and evolutionary strategies [CITE].
% While each flavor of an evolutionary algorithm possesses distinct characteristics that make it unique and applicable to certain problem domains, similarities can be found within each of them.
% Traditionally, evolutionary algorithms follow three key phases: evaluation, selection, and variation. 
% The framework developed in this thesis formally defines the selection scheme of an evolutionary algorithm, a component typically used during the selection phase of an evolutionary algorithm, into three components: population structure, trait construction, and selectors.
% Indeed, this framework allows us to apply tractable changes to a selection scheme and measure the impact the change has on problem-solving successes. 
% Additionally, this framework allows practitioners to easily identify the selection scheme within an evolutionary algorithm, while also helping reduce the likelihood of constructing redundant selection schemes.

% In Chapter 2, I focus on formalizing the selection scheme of an evolutionary algorithm, and in Chapters 3 and 4, I demonstrate how small alterations to a scheme can lead to different performances on the same optimization problem.
% In particular, lexicase selection [CITE] is the selection scheme that is extended through the lens of our selection scheme framework and assessed in Chapters 3 and 4.

% Ultimately, our new design of lexicase selection proved to be more effective than standard lexicase selection.




% When new evolutionary algorithms or components are developed, they are typically assessed through benchmark suites that contain optimization problems from different domains. 
% The latter allows practitioners to better understand the impact of the choices made for the configuration of an evolutionary algorithm has on its problem-solving success.
% Currently, the standard approach to evaluate an evolutionary algorithm's capabilities is through benchmark suites with differing optimization problems from different domains [CITE].


\input{01_introduction/sections/intro.tex}
\input{01_introduction/sections/lit_review_history.tex}
\input{01_introduction/sections/lit_review_cognition.tex}
\input{01_introduction/sections/thesis_work.tex}
% \section{Genetic programming}
% \section{Genetic algorithms}
% \section{Evolutionary programming}
% \section{Evolutionary strategies}
% \section{Similarities}
% \input{01_introduction/tex/selection-schemes}
% \input{01_introduction/tex/thesis-work}