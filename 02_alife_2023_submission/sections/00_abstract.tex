% 248/250 words used
% (250 words max, shorter preferred).

Due to the stochastic nature of evolution, not only is it hard to predict evolutionary outcomes, it is difficult to look at an evolved lineage and determine the key steps that pushed the population toward the final evolved state. 
Researchers have long examined the role of historical contingency in evolution; when do small, seemingly insignificant changes to a genotype substantially shift the probabilities of what traits or behaviors will ultimately evolve?
In recent decades, practitioners of experimental evolution have begun to investigate this question using a new technique: analytic replay experiments. 
By taking an evolved lineage and founding new evolving populations from various points along that lineage, we can measure any changes to the likelihood that a certain trait eventually evolves, known as the ``potentiation'' of that trait. 
Here we used digital organisms to conduct a high-resolution version of this technique.
We isolated how individual mutations altered the likelihood for learning or pre-learning strategies to evolve, with a focus on associative learning.
We find that the probability of evolving associative learning (i.e., its potentiation) can increase suddenly -- even with a single mutation that appeared innocuous when it occurred. 
While there was no obvious signal to identify potentiating mutations as they arose, we were able to retrospectively identify mechanisms by which these mutations influenced subsequent evolution.
Many of the most interesting and complex evolutionary adaptations that occur in nature are exceptionally rare.
Here, we extend techniques for understanding these rare evolutionary events and the patterns and processes that produce them.


% Scratch
%varied over lineages that successfully evolved associative learning in a digital evolution environment. 
%By leveraging a two-step analysis of lineages, we can zoom in to see the effect of individual mutations on potentiation. 
%This work demonstrates the potential of this technique in aiding our understanding of the evolution of intelligence in future studies. 