\section{Introduction}

% Lead in - how does this relate to the other chapters
Previous work has shown that the environment can influence the relative contributions of adaptation, chance, and history \cite{smithFitnessEvolvingBacterial2022}. 
I have previously proposed testing how changes to the environment and underlying representation affect potentiation in successful lineages (Chapters \ref{chap:replaying_associative_learning} and \ref{chap:simplified_model}). 
However, I will be unable to concretely identify if differences between those two systems are due to changing the environment or the representation. 
To remedy this issue, here I propose to quantify potentiation along successful lineages in two additional Avida environments.
By keeping the representation constant, potentiation differences must therefore be a result of the change in environment or the targeted trait within. %due to the one factor that has changed, the environment. 

% The results of Chapters \ref{chap:replaying_associative_learning} and \ref{chap:simplified_model} are uncertain, but I expect one of three possibilities: %can have many potential outcomes, but we can boil them down to three categories: 
% 1) we observe similar patterns in potentiation between the associative learning Avida environment and the simplified bitstring model, 
% 2) potentiation looks drastically different between the two setups, with no discernible patterns, or
% 3) a mix, where some traits are shared and others are drastically different. 
% My hypothesis is that the third option is what we will observe. 
% However, regardless of which situation is true, the results of those chapters will lay the initial expectations for how potentiation varies as systems change. 
% Between those chapters I proposed to vary both the environment and the representation of organisms. 
% Here, I propose to lock the representation, analyzing potentiation in two additional environments within Avida. 

%Regardless of which situation is true, the results of those chapters will prompt us to look at the similarities and differences in potentiation between the two systems. 
%If there are differences, can we untangle \textit{why} these differences exist? 
%If the patterns are mostly the same, can we use this information to predict how potentiation will change when looking at other environments and target behaviors?
%These are the questions I propose to investigate in this chapter by looking at two different environments and target behaviors in Avida. 

%Avida has been used to study a wide variety of evolutionary dynamics, ranging from X to Y to Z. 

% Why stick with Avida? Easier comparisons with the same levels of epistasis. 
I hypothesize that in many cases, potentiating mutations are highly epistatic, as epistasis is one key way for a seemingly innocuous mutation to have a drastic effect on later mutations.
As such, studies of potentiation should employ representations capable of epistatic interactions. 
This is an advantage of using Avida; there are multiple possibilities for epistasis between instructions, and in previous work I have confirmed that these interactions are sufficient for complex potentiation dynamics (Chapter \ref{chap:alife_submission}).
By conducting studies of potentiation in multiple Avida environments, I will investigate how environmental differences influence potentiation as the underlying representation, and thus the potential for epistatic interactions, only slightly differ. 

%We have demonstrated in Chapter \ref{chap:alife_submission} that we can quantify how the potentiation of a target behavior changes along a successful lineage in Avida.
%By the time we start on this project, we will have observed potentiation at a larger scale (Chapter \ref{chap:replaying_associative_learning}) and in a simplified model (Chapter \ref{chap:simplified_model}). 
Specifically, I propose to investigate potentiation in a patch harvesting behavior and in phenotypic plasticity. 
In the patch harvesting environment described in \citep{pontesEvolutionaryOriginsCognition2021}, I will target the successful harvesting of multiple patches in an organism's lifetime. %memory usage in the decision to exploit the current patch or explore to find a new patch. 
In the cyclic logic 6 environment (Chapter \ref{chap:consequences_of_plasticity}), I will target optimal plasticity. 
%By changing the environments and the behaviors we consider successful, we will be analyzing very different fitness landscapes and the effects they have on potentiation. 
While both target behaviors are complex, their details differ both with each other and with associative learning from previous work (Chapter \ref{chap:replaying_associative_learning}). 
Ultimately, will these details influence how potentiation changes? 

% Expectations and impact
%Specifically, here we target two different yet related behaviors. 
The patch harvesting behavior I am targeting is a basic cognitive behavior. 
In order to harvest multiple patches, organisms must switch between exploring to find new patches or exploiting the patch they are currently on. % state in order to successfully harvest multiple patches. 
Similar to the associative learning behavior in Chapter \ref{chap:replaying_associative_learning}, this behavior requires several different components in order to function (information storage, memory retrieval and usage, etc.).
As such, I expect the two behaviors to exhibit similar patterns in potentiation. 
%Specifically, I expect that single mutations (especially in the formation and usage of memory) can result in a drastic potentiation increase. 
The other behavior, optimal plasticity in the cyclic logic 6 environment, is a reactive behavior that does not require memory. % has similarities and differences to the other two behaviors.
It does, however, still consist of multiple components (performing tasks and regulating them due to environmental cues) that can be neutral or deleterious in isolation. 
Therefore, I also expect to see similar patterns in potentiation in the evolution of optimal plasticity. % regardless of the difference in the behaviors. 

% Conclusion - this is important because regardless of what the outcome is, the information we gain will be incredibly useful for future studies into potentiation and the role of history in evolution
This work will expand upon the previous chapters, providing additional examples of how potentiation changes as we vary the environment and target behavior. 
All together, these works will shape how we view potentiation, and will provide an expanding dataset for other researchers to compare against in the future. 
% Note that this will be the first time incorporating what we learned from chapter 4?

% As we discussed in the previous chapters, we can quantify how potentiation, the likelihood a focal trait evolves from a given genotype, changes over time. 
% In those chapters, I proposed studying these changes in potentiation in the context of associative learning (Chapters \ref{chap:alife_submission} and \ref{chap:replaying_associative_learning}) and in a simple bitstring NK landscape model (Chapter \ref{chap:simplified_model}). 
% The goal of those chapters is to try and find common trends in how potentiation changes across multiple lineages. 
% Here we continue to expand on these questions by looking at potentiation of different behaviors in two other environments: microbial mat harvesting seen by early bilaterian animals and the classic Logic 9 Avida environment. 

% This chapter asks if the target behavior influences how potentiation changes over time given a fixed substrate (in this case Avida organisms). 
% We will conduct replay experiments on successful lineages from both environments, applying insights learned from the bitstring model of Chapter \ref{chap:simplified_model}, to see if any general patterns emerge. 
% If we do not see general trends in the evolution of different behaviors in Avida, that will be strong evidence that potentiation is very unique to the target behavior we are analyzing. 
% If we \textit{do} see general trends, however, this work (along with the previous chapters) will lay the groundwork for future studies in other environments, computational substrates, or even in living organisms. 