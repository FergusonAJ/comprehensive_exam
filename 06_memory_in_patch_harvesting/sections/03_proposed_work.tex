\section{Proposed work}

In both environments, I will measure attributes of potentiation by conducting analytic replay experiments as described in Chapter \ref{chap:replaying_associative_learning}. 
I will then compare the measurements to previous data from Chapters \ref{chap:replaying_associative_learning} and \ref{chap:simplified_model} to determine what differences exist between the different environments. 
This work will provide additional examples of potentiation and allow the first cross-environment comparisons while keeping the underlying representation the same. 

Potentiation will be quantified using the exact same methods as Chapter \ref{chap:replaying_associative_learning}. 
Initial evolutionary replicates seeded with the default ancestor will be conducted to find replicates that exhibit the target behavior (multi-patch harvesting or optimal plasticity). 
For each environment, I will run initial replicates until I reach 40 successful lineages. 
To measure the changes in potentiation along the dominant lineage of each replicate, I will conduct two phases of replay experiments. 
The first will seed replicates with the genotypes found 50, 100, 150, and 200 steps before the target behavior first appeared in the lineage. 
I will continue to go backward along the lineage until a step shows less than 10 percentage points of improvement over the success rate from the initial replicates.
Once the exploratory replays have been conducted, I will identify potentiation windows (as in Chapter \ref{chap:replaying_associative_learning}) and seed replay replicates for every genotype in the windows. 
As before, I will run fifty replay replicates for each genotype analyzed. 

After conducting the replay experiments, I will collect the same potentiation measures described in Section \ref{sub:potentiation_measures}.
%These include maximum single-step potentiation gain/loss, the number of potentiation gain/loss windows, the fitness effect and behavioral background of potentiating/anti-potentiating mutations, and the distance between potentiating mutations and the appearance of the target behavior. 
Since these are the same measurements recorded in Chapters \ref{chap:replaying_associative_learning} and \ref{chap:simplified_model}, I can then statistically compare the distributions of each measurement across environments. 
The one exception is the behavioral background, which is unique to each task and thus can only be compared within a given environment. 
The cross-environment comparisons will be conducted first as a Kruskal-Wallis test to determine if significant differences exist across all of the environments \citep{kruskal_use_1952}. 
If a difference is detected, I will then conduct pairwise Mann-Whitney-Wilcoxon tests to look for significant differences between each pair of environments \citep{10.2307/3001968}. 
Finally, I will use a Holm-Bonferroni correction for multiple comparisons \citep{holmSimpleSequentiallyRejective1979}.

These comparisons will provide insight into how these potentiation dynamics vary when the environment changes but the representation stays the same. 
I expect only small differences between the associative learning behavior of Chapter \ref{chap:replaying_associative_learning} and the patch harvesting behavior examined here. 
Both cognitive behaviors require memory and have exist in environments containing non-cognitive alternative behaviors with high performance that might function as local optima. 
For the other environment, cyclic logic 6, I expect to see similar overall dynamics but a difference in the measured values. 
While the environment and behavior are drastically different from the other two, one particularly notable difference is that $>40\%$ of replicates evolved optimal plasticity in the different experiments of Chapter \ref{chap:consequences_of_plasticity}. 
Therefore I do not expect the same levels of potentiation gain as we observed in the evolution of associative learning in Chapter \ref{chap:alife_submission}, simply because the lineages start with a much higher level of potentiation. 
%Alternatively, lineages could still experience these  most of their potentiation in a single mutation, or we could observe more potentiation losses in this environment. 
Alternatively, we must consider that the cyclic logic 6 environment is the only proposed environment that changes with time; these changes to selective pressures may repeatedly break down potential building blocks and thus increase the potentiation gain when they are finally utilized. 
Regardless of the outcome, this work will provide needed data on potentiation and more insight into how evolution is (or is not!) contingent on initially innocuous mutations. 